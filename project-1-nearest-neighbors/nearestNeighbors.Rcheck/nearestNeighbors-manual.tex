\nonstopmode{}
\documentclass[a4paper]{book}
\usepackage[times,inconsolata,hyper]{Rd}
\usepackage{makeidx}
\usepackage[utf8]{inputenc} % @SET ENCODING@
% \usepackage{graphicx} % @USE GRAPHICX@
\makeindex{}
\begin{document}
\chapter*{}
\begin{center}
{\textbf{\huge Package `nearestNeighbors'}}
\par\bigskip{\large \today}
\end{center}
\begin{description}
\raggedright{}
\inputencoding{utf8}
\item[Type]\AsIs{Package}
\item[Title]\AsIs{Compute Nearest neighbors using C++}
\item[Version]\AsIs{2019.1.31}
\item[Author]\AsIs{Nicholas Anderson, Jacob Lemon, Megan Mikami}
\item[Maintainer]\AsIs{Nicholas Anderson }\email{nha4@nau.edu}\AsIs{}
\item[Description]\AsIs{Efficient computation of nearest neighbors}
\item[License]\AsIs{GPL-3}
\item[Encoding]\AsIs{UTF-8}
\item[LazyData]\AsIs{true}
\item[LinkingTo]\AsIs{RcppEigen}
\item[RoxygenNote]\AsIs{6.1.1}
\item[Imports]\AsIs{}
\item[Suggests]\AsIs{ElemStatLearn}
\item[NeedsCompilation]\AsIs{yes}
\item[Archs]\AsIs{i386, x64}
\end{description}
\Rdcontents{\R{} topics documented:}
\inputencoding{utf8}
\HeaderA{nn}{nearest neighbors algorithm}{nn}
%
\begin{Description}\relax
R function that wraps the C++ code.
\end{Description}
%
\begin{Usage}
\begin{verbatim}
nn(x.mat, y.vec, testx.vec, max.neighbors)
\end{verbatim}
\end{Usage}
%
\begin{Arguments}
\begin{ldescription}
\item[\code{x.mat}] numeric train featrue matrix [n x p]

\item[\code{y.vec}] numeric train label vecotr [n], either all 0/1 for binary classificiation, or other real numbers for regression (multi-class classification not supported)

\item[\code{testx.vec}] numeric test feature vector [p]

\item[\code{max.neighbors}] scalar integer, max number of neighbors
\end{ldescription}
\end{Arguments}
%
\begin{Value}
numeric vector of size man.neighbors, predictions from 1 to max.neighbors
\end{Value}
%
\begin{Examples}
\begin{ExampleCode}
data(zip.train, package-"ElemStatLearn")
i01 <- which(zip.train[,1] %in% c(0,1))
train.i <- i01[1:5]
test.i <- i01[6]
x <- zip.train[train.i, -1]
y <- zip.train[train.i, 1]
testx <- zip.train[test.i, -1]
nn(x ,y, testx , 3)
zip.train[test.i, 1]

\end{ExampleCode}
\end{Examples}
\printindex{}
\end{document}
